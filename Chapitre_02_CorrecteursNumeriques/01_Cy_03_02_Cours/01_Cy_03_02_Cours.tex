%%%% Paramétrage du cours %%%%
\def\xxactivite{Cours}
\def\xxauteur{\textsl{Xavier Pessoles}}

\fichefalse \proftrue \tdfalse \courstrue

\def\xxnumchapitre{Chapitre 2 \vspace{.2cm}}
\def\xxchapitre{\hspace{.12cm} Correction numérique des systèmes asservis}

\def\xxcompetences{%
\textsl{%
\textbf{Savoirs et compétences :}\\
%\begin{itemize}[label=\ding{112},font=\color{ocre}] 
%\item \textit{Res1.C4 : } correction;
%\item \textit{Res1.C4.SF1 : } proposer la démarche de réglage d’un correcteur proportionnel, proportionnel intégral et à avance de phase,
%\item \textit{Con.C2 : } 	correction d’un système asservi	;
%\item \textit{Con.C2.SF1 : } choisir un type de correcteur adapté.
%\end{itemize}
}}


\def\xxfigures{
%\includegraphics[width=3cm]{SoloWheel_Orbit}%
%\\
%\textit{SoloWheel Orbit.}
}%figues de la page de garde


\input{\repStyle/new_pagegarde}

\setlength{\columnseprule}{.1pt}

\vspace{2cm}
\pagestyle{fancy}
\thispagestyle{plain}

%%%%%%%%%%%%%%%%%%%%%%%

\section{Rappels sur les caractéristiques des signaux numériques}
Lors du traitement des signaux par un ordinateur, un microcôntroleur, un automate ou autre dispositif, les signaux analogiques (continues) sont convertis en signaux numériques (discrets) pour être traités. On parle de conversion analogique -- numérique (CAN). 
L'opération inverse est alors réalisée lorsqu'il est nécessaire de restituer de l'information à l'utilisateur.
\begin{center}
\includegraphics[width=.8\linewidth]{CAN_CNA}
\end{center}

\subsection{Echantillonnage et quantification}

\begin{defi}{Echantillonnage}
Échantillonner un signal consiste à prélever les valeurs de ce signal à intervalles définis. Si ces intervalles sont réguliers, on note $T_e$ la période d'échantillonnage (en seconde), c'est-à-dire la durée entre deux prélèvements. On note $f_e = \dfrac{1}{T_e}$ la fréquence d'écantillonnage (en Hertz), c'est-à-dire le nombre d'échantillons par seconde.

Si on note e(t) le signal continu, on notera $e_k = e(kT_e)$ la valeur de l'échantillon $e(t)$ à l'instant $kT_e$. 

\end{defi}

*** Choix de la période d'échantillonnage***



\begin{theorem}{Théorème de Shannon}
Soit un signal périodique décomposable en signaux périodiques dont la fréquence maximale présente est largement supérieure à celle minimale présente.
La représentation discrète d’un signal par des échantillons régulièrement espacés exige une fréquence d’échantillonnage supérieure au double de la fréquence maximale présente dans ce signal.
\end{theorem}
*** Conséquences du théorème Shannon *** 



\begin{defi}{Quantification}
Quantifier in signal consiste à approcher un signal continu par les valeurs d'un ensemble discret. 

En général, le système étant échantilloné sur un système binaire sur $N$ bits, on peut donc approcher le signal par $2^N$ valeurs discrètes.
\end{defi}

\begin{defi}{Erreur de quantification}
Soit un signal continu borné entre les valeurs $e_{\text{min}}$ et $e_{\text{max}}$ échantilloné sur $N$ bits. Le pas de quantification donne aussi l'erreur maximala de quantification. On a $q = \dfrac{e_{\text{max}}-e_{\text{min}}}{2^N}$.

\end{defi}


\subsection{Conversion analogique -- numérique}

\subsection{Conversion numérique -- analogique}

\section{Filtrage numérique}
\begin{defi}{Filtrage numérique}~\\
\begin{minipage}[c]{.7\linewidth}
Soit $e_k$ un signal discret. Soit $s_k$ un signal discret filtré. $k\in \mathbb{N}$ est le numéro de l'échantillon. On appelle filtrage numérique l'opération mathématique permettant de filtrer un signal, c'est-à-dire d'éliminer certaines composantes harmoniques. 
\end{minipage} \hfill
\begin{minipage}[c]{.25\linewidth}
\begin{center}
\includegraphics[width=\linewidth]{fig_01}
\end{center}
\end{minipage}
\end{defi}

\subsection{Filtrage numérique passe-bas}
\begin{defi}{Filtre passe-bas du premier ordre}~\\
On donne l'équation différentielle d'un filtre du premier ordre : $s(t)+\tau \deriv{s(t)}{} = e(t)$.  $T_e$ est la période d'échantillonnage du système. La pulsation de coupure à $-\SI{3}{dB}$ du filtre est donné par $\omega_c = \dfrac{1}{\tau}$.

On montre que $s_k = \dfrac{T_e}{T_e+\tau} e_k + \dfrac{\tau}{T_e+\tau} s_{k-1}$. 
\end{defi}

\begin{demo}
On a  $s(t)+\tau \deriv{s(t)}{} = e(t)$. En approximant $\deriv{s(t)}{}$ par $\dfrac{s(t) -s(t-T_e)}{T_e}$, puis en discrétisant l'équation différentielle, on a donc : 
 $s_k +\tau \dfrac{s_k - s_{k-1}}{T_e} = e_k$
 $\Leftrightarrow s_k \left(1+\dfrac{\tau}{T_e}\right) - s_{k-1}\dfrac{\tau}{T_e} = e_k$
 $\Leftrightarrow s_k \dfrac{T_e+\tau}{T_e}  = e_k + s_{k-1}\dfrac{\tau}{T_e}$
  $\Leftrightarrow s_k   = e_k \dfrac{T_e}{T_e+\tau}+ s_{k-1}\dfrac{\tau}{T_e}\dfrac{T_e}{T_e+\tau}$
   $\Leftrightarrow s_k   = e_k \dfrac{T_e}{T_e+\tau}+ s_{k-1}\dfrac{\tau}{T_e+\tau}$.
\end{demo}

\subsection{Filtrage numérique à moyenne glissante} 

\section{Correcteurs numériques}
 
\subsection{Correcteur P}
\begin{defi}{Correcteur proportionnel}
Dans le domaine temporel, on a $u(t)=K_P \varepsilon(t)$. Cela est traduit par le relation de récurrence suivante : 
$u_k = K_p \varepsilon_k$.
\end{defi}
\subsection{Correcteur PI}
\begin{defi}{Correcteur proportionnel intégral}
Dans le domaine temporel, on a $u(t)=K_P  \left(\varepsilon(t)+\dfrac{1}{T_i}\int\limits_0^t \varepsilon(\tau) \dd \tau \right)$. Cela est traduit par le relation de récurrence suivante : 
$u_k = u_{k-1} +  K_p \left( \varepsilon_k \left( 1+\dfrac{T_e}{T_i}\right) - \varepsilon_{k-1} \right)$.
\end{defi}

\subsection{Correcteur PID}
\begin{defi}{Correcteur proportionnel intégral et dérivé}

Dans le domaine temporel, on a $u(t)=K_P  \left(\varepsilon(t)+\dfrac{1}{T_i}\int\limits_0^t \varepsilon(\tau) \dd \tau +T_d \dfrac{\dd \varepsilon(t)}{\dd t}\right)$. Cela est traduit par le relation de récurrence suivante : 
$u_k = u_{k-1} + K_p \left( \varepsilon_k \left( 1+\dfrac{T_e}{T_i}+\dfrac{T_d}{T_e}\right) - \varepsilon_{k-1} \left( 1+\dfrac{T_d}{T_e}\right)\right)$.
\end{defi}


\section{Limite des correcteurs lors de la commande de systèmes}

\begin{thebibliography}{2}
%   \bibitem[1]{ref1} Frédéric Mazet, {\it Cours d'automatique de deuxième année, Lycée Dumont Durville, Toulon.}
%   \bibitem[2]{ref2} Florestan Mathurin, {\it Correction des SLCI, Lycée Bellevue, Toulouse, \url{http://florestan.mathurin.free.fr/}.}
%   \bibitem[3]{ref3} Damien Iceta, David Violeau, Alain Caignot, Xavier Pessoles, Vincent Boyer, François Golanski, {\it Sciences industrielles de l'ingénieur MP/MP* PSI/PSI* PT/PT*, Méthodes. Exercices. Problèmes. Sujets de concours. Vuibert Prépas.}
\end{thebibliography}





