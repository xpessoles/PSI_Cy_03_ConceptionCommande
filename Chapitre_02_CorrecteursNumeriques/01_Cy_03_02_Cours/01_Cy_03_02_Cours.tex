%%%% Paramétrage du cours %%%%
\def\xxactivite{Cours}
\def\xxauteur{\textsl{Xavier Pessoles}}

\fichefalse \proftrue \tdfalse \courstrue

\def\xxnumchapitre{Chapitre 2 \vspace{.2cm}}
\def\xxchapitre{\hspace{.12cm} Correction numérique des systèmes asservis}

\def\xxcompetences{%
\textsl{%
\textbf{Savoirs et compétences :}\\
%\begin{itemize}[label=\ding{112},font=\color{ocre}] 
%\item \textit{Res1.C4 : } correction;
%\item \textit{Res1.C4.SF1 : } proposer la démarche de réglage d’un correcteur proportionnel, proportionnel intégral et à avance de phase,
%\item \textit{Con.C2 : } 	correction d’un système asservi	;
%\item \textit{Con.C2.SF1 : } choisir un type de correcteur adapté.
%\end{itemize}
}}


\def\xxfigures{
%\includegraphics[width=3cm]{SoloWheel_Orbit}%
%\\
%\textit{SoloWheel Orbit.}
}%figues de la page de garde


\input{\repStyle/new_pagegarde}

\setlength{\columnseprule}{.1pt}

\vspace{2cm}
\pagestyle{fancy}
\thispagestyle{plain}

%%%%%%%%%%%%%%%%%%%%%%%

\section{Rappels sur les caractéristiques des signaux numériques}

\subsection{Echantillonnage et quantification}
\begin{defi}{Echantillonnage}
\end{defi}
\begin{defi}{Quantification}
\end{defi}

\begin{theorem}{Théorème de Shannon}
\end{theorem}

\subsection{Conversion analogique -- numérique}

\subsection{Conversion numérique -- analogique}

\section{Filtrage numérique}
 

\section{Correcteurs numériques}
 
\subsection{Correcteur P}
\begin{defi}{Correcteur proportionnel}
Dans le domaine temporel, on a $u(t)=K_P \varepsilon(t)$. Cela est traduit par le relation de récurrence suivante : 
$u_k = K_p \varepsilon_k$.
\end{defi}
\subsection{Correcteur PI}
\begin{defi}{Correcteur proportionnel intégral}
Dans le domaine temporel, on a $u(t)=K_P  \left(\varepsilon(t)+\dfrac{1}{T_i}\int\limits_0^t \varepsilon(\tau) \dd \tau \right)$. Cela est traduit par le relation de récurrence suivante : 
$u_k = u_{k-1} K_p \left( \varepsilon_k \left( 1+\dfrac{T_e}{T_i}\right) - \varepsilon_{k-1} \right)$.
\end{defi}

\subsection{Correcteur PID}

\section{Limite des correcteurs lors de la commande de systèmes}

\begin{thebibliography}{2}
%   \bibitem[1]{ref1} Frédéric Mazet, {\it Cours d'automatique de deuxième année, Lycée Dumont Durville, Toulon.}
%   \bibitem[2]{ref2} Florestan Mathurin, {\it Correction des SLCI, Lycée Bellevue, Toulouse, \url{http://florestan.mathurin.free.fr/}.}
%   \bibitem[3]{ref3} Damien Iceta, David Violeau, Alain Caignot, Xavier Pessoles, Vincent Boyer, François Golanski, {\it Sciences industrielles de l'ingénieur MP/MP* PSI/PSI* PT/PT*, Méthodes. Exercices. Problèmes. Sujets de concours. Vuibert Prépas.}
\end{thebibliography}





