\documentclass[10pt,fleqn]{article} % Default font size and left-justified equations
\usepackage[%
    pdftitle={Correction des SLCI},
    pdfauthor={Xavier Pessoles}]{hyperref}
    
\input{style/new_style}
\input{style/macros_SII}
\usepackage{multicol}
\usepackage{siunitx}
%\usepackage{picins}
\fichetrue
%\fichefalse

\proftrue
\proffalse

\tdtrue
%\tdfalse

\courstrue
\coursfalse

\def\discipline{Sciences \\Industrielles de \\ l'Ingénieur}
\def\xxtete{Sciences Industrielles de l'Ingénieur}

\def\classe{PSI$\star$ -- MP}
\def\xxnumpartie{Cycle 03}
\def\xxpartie{Concevoir la partie commande des systèmes asservis afin de valider leurs performances}


\def\xxnumchapitre{Chapitre 1 \vspace{.2cm}}
\def\xxchapitre{\hspace{.12cm} Correction des SLCI}


\def\xxtitreexo{Colle 4}
\def\xxsourceexo{\hspace{.2cm} \footnotesize{Pôle Chateaubriand -- Joliot Curie}}


\def\xxposongletx{2}
\def\xxposonglettext{1.45}
\def\xxposonglety{20}
%\def\xxonglet{Part. 1 -- Ch. 3}
\def\xxonglet{Cycle 03}

\def\xxactivite{Colle 04}
\def\xxauteur{\textsl{Pôle Chateaubriand -- Joliot Curie}}

\def\xxcompetences{%
\textsl{%
\textbf{Savoirs et compétences :}\\
%Les sources sont associées par un \emph{hacheur série}. La détermination des grandeurs électriques associées à ce montage permet de conclure vis à vis du cahier des charges.
%\noindent \textbf{Résoudre :} à partir des modèles retenus :
%\begin{itemize}[label=\ding{112},font=\color{ocre}] 
%\item choisir une méthode de résolution analytique, graphique, numérique;
%\item mettre en \oe{}uvre une méthode de résolution.
%\end{itemize}
%\begin{itemize}[label=\ding{112},font=\color{ocre}] 
%\item \textit{Rés -- C1.1 :} Loi entrée sortie géométrique et cinématique -- Fermeture géométrique.
%\end{itemize}
%
%\noindent \textit{Mod2 -- C4.1 :} Représentation par schéma bloc.
}}

\def\xxfigures{
%\includegraphics[width=.5\linewidth]{images/fig_01}
}%figues de la page de garde


\def\xxpied{%
Cycle 03 -- Concevoir la partie commande des SLCI \\
Chapitre 1 -- \xxactivite%
}

\setcounter{secnumdepth}{5}
%---------------------------------------------------------------------------

\usepackage{pgfplots}
\begin{document}

%\chapterimage{png/Fond_Cin}
\input{style/new_pagegarde}
\vspace{5cm}
\pagestyle{fancy}
\thispagestyle{plain}

\def\columnseprulecolor{\color{ocre}}
\setlength{\columnseprule}{0.4pt} 

\def\pathfig{images}

\begin{multicols}{2}
\subsection*{Correction proportionnelle}
Soit $F(p)$ la FTBO d'un système bouclé à retour unitaire. Les diagrammes de BODE de $F(p)$ sont représentés
sur la figure ci-dessous.

\subparagraph{}\textit{Déterminer les marges de phase et de gain du système, puis conclure quant à sa stabilité.}


On décide d’ajouter au système un correcteur série de type proportionnel. On note $K_p$ le gain de ce
correcteur.

\subparagraph{}\textit{Déterminer la valeur de $K_p$ permettant d’obtenir une marge de gain $M_G =\SI{12}{dB}$.}

\subparagraph{}\textit{Déterminer la nouvelle marge de phase du système.}

\subparagraph{}\textit{En le justifiant, déterminer l’erreur de position du système corrigé pour une consigne indicielle.}


\subsection*{Correction intégrale -- Asservissement en accélération}
\setcounter{exo}{0}
On désire contrôler l'accélération $\gamma(t)$ d'un plateau. Pour cela, un capteur d'accélération, fixé sur le plateau
et de sensibilité $B$, est utilisé dans la chaîne de retour du système. Le moteur permettant la motorisation du
plateau est modélisé par la fonction de transfert : $H(s)=\dfrac{A}{1+\tau s}$. 
On modélise le correcteur par la fonction de transfert $C(s)$.
\begin{center}
\includegraphics[width=\linewidth]{images/fig_02}
\end{center}

On a $A=\SI{100}{g.m.s^{-2}.V^{-1}}$, $\tau=\SI{0,2}{s}$ et $B=10^{-2}\,\text{g}^{-1}\text{V}\text{m}^{-1}\text{s}^{-2}$.

\subparagraph{}\textit{Quelle doit être la fonction de transfert du transducteur $T(s)$ qui traduira l’accélération de consigne $\Gamma_c(s)$ en tension $E(s)$.}

On applique à l’entrée du système une consigne d’accélération $\gamma_c=20 g$.

Système asservi sans correction : $C(s)=1$.
\subparagraph{}\textit{Déterminer l'expression de la fonction de transfert en boucle fermée de ce système. Identifier les différents paramètres de cette fonction. Réaliser l'application numérique.}

\subparagraph{}\textit{Calculer le temps de réponse à 5\% de ce système pour une entrée en échelon.}

\subparagraph{}\textit{Donner la valeur de l'accélération en régime permanent. Ce système est-il précis ? Donner l'erreur en régime permanent.}

\subparagraph{}\textit{Donner l'allure de la réponse de ce système en précisant les points caractéristiques.}

Système asservi avec correction intégrale : $C(s)=\dfrac{1}{s}$.

\subparagraph{}\textit{Déterminer l'expression de la fonction de transfert en boucle fermée de ce système. Identifier les
différents paramètres de cette fonction. Réaliser l'application numérique.}

\subparagraph{}\textit{Calculer le temps de réponse à 5\% de ce système pour une entrée en échelon.}

\subparagraph{}\textit{Donner la valeur de l'accélération en régime permanent. Ce système est-il précis ? Donner l'erreur en
régime permanent. Pouvait-on prévoir ce résultat.}

\subparagraph{}\textit{Conclure en comparant le comportement du système avec et sans correction.}

%\subparagraph{}\textit{}

\end{multicols}

%1.1
%2.1
\begin{center}
\includegraphics[width=\linewidth]{images/fig_01}

%\includegraphics[width=\linewidth]{images/cor_02}
\end{center}

\newpage
\begin{center}
\includegraphics[width=\linewidth]{images/cor_01}

\includegraphics[width=\linewidth]{images/cor_02}
\end{center}
\end{document}