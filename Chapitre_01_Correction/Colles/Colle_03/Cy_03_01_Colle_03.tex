\documentclass[10pt,fleqn]{article} % Default font size and left-justified equations
\usepackage[%
    pdftitle={Correction des SLCI},
    pdfauthor={Xavier Pessoles}]{hyperref}
    
\input{style/new_style}
\input{style/macros_SII}
\usepackage{multicol}
\usepackage{siunitx}
%\usepackage{picins}
\fichetrue
%\fichefalse

\proftrue
\proffalse

\tdtrue
%\tdfalse

\courstrue
\coursfalse

\def\discipline{Sciences \\Industrielles de \\ l'Ingénieur}
\def\xxtete{Sciences Industrielles de l'Ingénieur}

\def\classe{PSI$\star$ -- MP}
\def\xxnumpartie{Cycle 03}
\def\xxpartie{Concevoir la partie commande des systèmes asservis afin de valider leurs performances}


\def\xxnumchapitre{Chapitre 1 \vspace{.2cm}}
\def\xxchapitre{\hspace{.12cm} Correction des SLCI}


\def\xxtitreexo{Colle 3}
\def\xxsourceexo{\hspace{.2cm} \footnotesize{Équipe PT La Martinière}}


\def\xxposongletx{2}
\def\xxposonglettext{1.45}
\def\xxposonglety{20}
%\def\xxonglet{Part. 1 -- Ch. 3}
\def\xxonglet{Cycle 03}

\def\xxactivite{Colle 03}
\def\xxauteur{\textsl{Équipe PT La Martinière}}

\def\xxcompetences{%
\textsl{%
\textbf{Savoirs et compétences :}\\
%Les sources sont associées par un \emph{hacheur série}. La détermination des grandeurs électriques associées à ce montage permet de conclure vis à vis du cahier des charges.
%\noindent \textbf{Résoudre :} à partir des modèles retenus :
%\begin{itemize}[label=\ding{112},font=\color{ocre}] 
%\item choisir une méthode de résolution analytique, graphique, numérique;
%\item mettre en \oe{}uvre une méthode de résolution.
%\end{itemize}
%\begin{itemize}[label=\ding{112},font=\color{ocre}] 
%\item \textit{Rés -- C1.1 :} Loi entrée sortie géométrique et cinématique -- Fermeture géométrique.
%\end{itemize}
%
%\noindent \textit{Mod2 -- C4.1 :} Représentation par schéma bloc.
}}

\def\xxfigures{
%\includegraphics[width=.5\linewidth]{images/fig_01}
}%figues de la page de garde


\def\xxpied{%
Cycle 03 -- Concevoir la partie commande des SLCI \\
Chapitre 1 -- \xxactivite%
}

\setcounter{secnumdepth}{5}
%---------------------------------------------------------------------------

\usepackage{pgfplots}
\begin{document}

%\chapterimage{png/Fond_Cin}
\input{style/new_pagegarde}
\vspace{5cm}
\pagestyle{fancy}
\thispagestyle{plain}

\def\columnseprulecolor{\color{ocre}}
\setlength{\columnseprule}{0.4pt} 

\def\pathfig{images}

\begin{multicols}{2}




On considère un système de fonction de transfert est :  $G(p)=\dfrac{K}{(p+1)^3}$ placé dans une boucle de régulation à retour unitaire. On souhaite une marge de phase supérieure à 45\degres.

\subparagraph{}
\textit{Définir la condition de stabilité théorique du système ? }


On note $t_m$ le temps de montée du système en BF et $t_m\simeq \dfrac{3}{\omega_{\text{co}}}$ et $\omega_{\text{co}}$ est la pulsation de coupure à \SI{0}{dB} du système en BO.  


\subparagraph{}
\textit{Calculer la valeur $K$ qui assure, en boucle fermée, un temps de montée de \SI{2,15}{s}.}

\subparagraph{}
\textit{Calculer pour cette valeur de $K$ la marge de phase.}

\subparagraph{}
\textit{En déduire l'expression de la fonction de transfert du correcteur à avance de phase $C(p)=K_a\dfrac{1+aTp}{1+Tp}$ qu'il faut introduire dans la chaîne directe.  }




\end{multicols}
\newpage
\begin{center}
\includegraphics[width=\linewidth]{images/cor_01}

%\includegraphics[width=\linewidth]{images/cor_02}
\end{center}
\end{document}