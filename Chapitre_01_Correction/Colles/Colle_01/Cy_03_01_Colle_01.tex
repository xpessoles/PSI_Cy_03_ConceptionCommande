\documentclass[10pt,fleqn]{article} % Default font size and left-justified equations
\usepackage[%
    pdftitle={Correction des SLCI},
    pdfauthor={Xavier Pessoles}]{hyperref}
    
\input{style/new_style}
\input{style/macros_SII}
\usepackage{multicol}
\usepackage{siunitx}
%\usepackage{picins}
\fichetrue
%\fichefalse

\proftrue
\proffalse

\tdtrue
%\tdfalse

\courstrue
\coursfalse

\def\discipline{Sciences \\Industrielles de \\ l'Ingénieur}
\def\xxtete{Sciences Industrielles de l'Ingénieur}

\def\classe{PSI$\star$ -- MP}
\def\xxnumpartie{Cycle 03}
\def\xxpartie{Concevoir la partie commande des systèmes asservis afin de valider leurs performances}


\def\xxnumchapitre{Chapitre 1 \vspace{.2cm}}
\def\xxchapitre{\hspace{.12cm} Correction des SLCI}


\def\xxtitreexo{Four pour traitement thermique}
\def\xxsourceexo{\hspace{.2cm} \footnotesize{Équipe PT La Martinière}}


\def\xxposongletx{2}
\def\xxposonglettext{1.45}
\def\xxposonglety{20}
%\def\xxonglet{Part. 1 -- Ch. 3}
\def\xxonglet{Cycle 02}

\def\xxactivite{Colle 01}
\def\xxauteur{\textsl{Équipe PT La Martinière}}

\def\xxcompetences{%
\textsl{%
\textbf{Savoirs et compétences :}\\
%Les sources sont associées par un \emph{hacheur série}. La détermination des grandeurs électriques associées à ce montage permet de conclure vis à vis du cahier des charges.
%\noindent \textbf{Résoudre :} à partir des modèles retenus :
%\begin{itemize}[label=\ding{112},font=\color{ocre}] 
%\item choisir une méthode de résolution analytique, graphique, numérique;
%\item mettre en \oe{}uvre une méthode de résolution.
%\end{itemize}
%\begin{itemize}[label=\ding{112},font=\color{ocre}] 
%\item \textit{Rés -- C1.1 :} Loi entrée sortie géométrique et cinématique -- Fermeture géométrique.
%\end{itemize}
%
%\noindent \textit{Mod2 -- C4.1 :} Représentation par schéma bloc.
}}

\def\xxfigures{
%\includegraphics[width=.5\linewidth]{images/fig_01}
}%figues de la page de garde


\def\xxpied{%
Cycle 03 -- Concevoir la partie commande des SLCI \\
Chapitre 1 -- \xxactivite%
}

\setcounter{secnumdepth}{5}
%---------------------------------------------------------------------------

\usepackage{pgfplots}
\begin{document}

%\chapterimage{png/Fond_Cin}
\input{style/new_pagegarde}
\vspace{5cm}
\pagestyle{fancy}
\thispagestyle{plain}

\def\columnseprulecolor{\color{ocre}}
\setlength{\columnseprule}{0.4pt} 

\def\pathfig{images}

\begin{multicols}{2}
Un four électrique destiné au traitement thermique d'objets est constitué d'une enceinte close chauffée par une résistance électrique alimentée par une tension v(t). Dix objets peuvent prendre place simultanément dans le four. Le traitement thermique consiste à maintenir les objets pendant 1 heure à une température de 1200\degres C (régulée de façon optimale car les objets sont détruits si la température dépasse  1400\degres C).
Entre deux cuissons, un temps de 24 minutes est nécessaire pour procéder au refroidissement du four et à la manutention.
Le four est régi par l’équation différentielle : $\dfrac{\text{d}\theta(t)}{\text{d}t}+2000\dfrac{\text{d}^2\theta(t)}{\text{d}t^2}=0,02 v(t)$.


\subparagraph{}\textit{Calculer la fonction de transfert $G(p)$ du four en boucle ouverte. Quel est le gain statique du four? Que se passerait-il si on alimentait le four en continu et en boucle ouverte ?}

On décide de réguler la température $\theta(t)$ dans le four en utilisant un capteur de température qui délivre une tension $u(t)$. Le capteur est régi par l’équation différentielle :  $u(t)+2\dfrac{\text{d}u(t)}{\text{d}t}=5\cdot 10^{-3} \theta(t)$. On introduit également un gain $K$ dans la chaîne directe.

\subparagraph{}\textit{Faire le schéma de la boucle de régulation et calculer sa fonction de transfert en boucle fermée. Rappeler  les conditions de stabilité d'un système.}

\subparagraph{}\textit{On souhaite se placer dans des conditions de stabilité suffisantes en imposant une marge de phase $\Delta \varphi= 45\degres$. Quelle  est dans ces conditions, la valeur du temps de montée en boucle fermée (voir ci-dessous) ?}

On donne $t_m$ le temps de montée du système en BF et $t_m\simeq \dfrac{3}{\omega_{\text{co}}}$ et $\omega_{\text{co}}$ est la pulsation de coupure à \SI{0}{dB} du système en BO.  

On souhaite atteindre une cadence de 100 pièces en 24h, ceci est obtenu pour $K=11,3$.

\subparagraph{}
\textit{Pour conserver une marge de phase égale à 60° on introduit une correcteur à avance de phase sous la forme $C(p)=K_a \dfrac{1+aTp}{1+Tp}$. Déterminer les constantes du correcteur.}



\end{multicols}
\newpage
\begin{center}
\includegraphics[width=\linewidth]{images/cor_01}
\end{center}
\end{document}