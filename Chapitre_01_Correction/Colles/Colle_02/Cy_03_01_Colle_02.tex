\documentclass[10pt,fleqn]{article} % Default font size and left-justified equations
\usepackage[%
    pdftitle={Correction des SLCI},
    pdfauthor={Xavier Pessoles}]{hyperref}
    
\input{style/new_style}
\input{style/macros_SII}
\usepackage{multicol}
\usepackage{siunitx}
%\usepackage{picins}
\fichetrue
%\fichefalse

\proftrue
\proffalse

\tdtrue
%\tdfalse

\courstrue
\coursfalse

\def\discipline{Sciences \\Industrielles de \\ l'Ingénieur}
\def\xxtete{Sciences Industrielles de l'Ingénieur}

\def\classe{PSI$\star$ -- MP}
\def\xxnumpartie{Cycle 03}
\def\xxpartie{Concevoir la partie commande des systèmes asservis afin de valider leurs performances}


\def\xxnumchapitre{Chapitre 1 \vspace{.2cm}}
\def\xxchapitre{\hspace{.12cm} Correction des SLCI}


\def\xxtitreexo{Colle 2}
\def\xxsourceexo{\hspace{.2cm} \footnotesize{Équipe PT La Martinière}}


\def\xxposongletx{2}
\def\xxposonglettext{1.45}
\def\xxposonglety{20}
%\def\xxonglet{Part. 1 -- Ch. 3}
\def\xxonglet{Cycle 02}

\def\xxactivite{Colle 02}
\def\xxauteur{\textsl{Équipe PT La Martinière}}

\def\xxcompetences{%
\textsl{%
\textbf{Savoirs et compétences :}\\
%Les sources sont associées par un \emph{hacheur série}. La détermination des grandeurs électriques associées à ce montage permet de conclure vis à vis du cahier des charges.
%\noindent \textbf{Résoudre :} à partir des modèles retenus :
%\begin{itemize}[label=\ding{112},font=\color{ocre}] 
%\item choisir une méthode de résolution analytique, graphique, numérique;
%\item mettre en \oe{}uvre une méthode de résolution.
%\end{itemize}
%\begin{itemize}[label=\ding{112},font=\color{ocre}] 
%\item \textit{Rés -- C1.1 :} Loi entrée sortie géométrique et cinématique -- Fermeture géométrique.
%\end{itemize}
%
%\noindent \textit{Mod2 -- C4.1 :} Représentation par schéma bloc.
}}

\def\xxfigures{
%\includegraphics[width=.5\linewidth]{images/fig_01}
}%figues de la page de garde


\def\xxpied{%
Cycle 03 -- Concevoir la partie commande des SLCI \\
Chapitre 1 -- \xxactivite%
}

\setcounter{secnumdepth}{5}
%---------------------------------------------------------------------------

\usepackage{pgfplots}
\begin{document}

%\chapterimage{png/Fond_Cin}
\input{style/new_pagegarde}
\vspace{5cm}
\pagestyle{fancy}
\thispagestyle{plain}

\def\columnseprulecolor{\color{ocre}}
\setlength{\columnseprule}{0.4pt} 

\def\pathfig{images}

\begin{multicols}{2}
On considère un système de fonction de transfert en boucle ouverte $G(p)$ que l'on souhaite réguler à l’aide d'une boucle à retour unitaire : $G(p)=\dfrac{K}{\left(10p+1 \right)^2\left(p+1 \right)}$

On souhaite que la boucle de régulation fonctionne selon le cahier des charges suivant :
\begin{itemize}
\item marge de phase : $\Delta \varphi \geq 45\degres$;
\item dépassement $D\% < 10\%$ ; 
\item écart statique $\varepsilon_S < 0,08$ ; 
\item temps de montée $t_m < \SI{8}{s}$.
\end{itemize}

\subparagraph{}
\textit{Quelle est la condition sur $K$ pour obtenir $\varepsilon_S<0,08$ ?}

On note $t_m$ le temps de montée du système en BF et $t_m\simeq \dfrac{3}{\omega_{\text{co}}}$ et $\omega_{\text{co}}$ est la pulsation de coupure à \SI{0}{dB} du système en BO.  

\subparagraph{}
\textit{Quelle est la condition sur $K$ pour obtenir $t_m<\SI{8}{s}$ ?}


\subparagraph{}
\textit{Quel choix faire pour la valeur de $K$ ?}

\subparagraph{}
\textit{Calculer la valeur de la marge de phase obtenue dans ces conditions. }


Expérimentalement, on constate que 
$z_{\text{BF}}\simeq \dfrac{\Delta \varphi ^{o}}{100}$ 
et on rappelle que $D\% = e^{\dfrac{-\pi z_{BF}}{\sqrt{1-z_{BF}^2}}}$.

\subparagraph{}
\textit{Que vaut alors le dépassement D\%?}


\subparagraph{}
\textit{À partir de la relation précédente, déterminer la marge de phase qui correspond à un dépassement de 10\%.}

Avec la valeur de $K=16,1$, on introduit, en amont de $G(p)$, dans la chaîne directe, un correcteur $C(p)=K_a \dfrac{1+aTp}{1+Tp}$ à avance de phase destiné à corriger le dépassement et la marge de phase, sans altérer ni la rapidité, ni la précision qui correspondent au cahier des charges.

\subparagraph{}
\textit{Déterminer alors la fonction de transfert de ce correcteur à avance de phase permettant d’obtenir une marge de phase de 60\degres.}

\end{multicols}
\newpage
\begin{center}
\includegraphics[width=\linewidth]{images/cor_01}

\includegraphics[width=\linewidth]{images/cor_02}
\end{center}
\end{document}