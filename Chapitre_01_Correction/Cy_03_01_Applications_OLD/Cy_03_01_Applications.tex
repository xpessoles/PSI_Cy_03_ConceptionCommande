\documentclass[10pt,fleqn]{article} % Default font size and left-justified equations
\usepackage[%
    pdftitle={Modélisation SLCI : Stabilité des systèmes},
    pdfauthor={Xavier Pessoles}]{hyperref}
    
\input{style/new_style}
\input{style/macros_SII}
\usepackage{multicol}
\usepackage{siunitx}
%\usepackage{picins}
\fichetrue
%\fichefalse

\proftrue
\proffalse

\tdtrue
%\tdfalse

\courstrue
\coursfalse

\def\discipline{Sciences \\Industrielles de \\ l'Ingénieur}
\def\xxtete{Sciences Industrielles de l'Ingénieur}

\def\classe{PSI$\star$ -- MP}
\def\xxnumpartie{Cycle 03}
\def\xxpartie{Concevoir la partie commande des systèmes asservis afin de valider leurs performances}


\def\xxnumchapitre{Chapitre 1 \vspace{.2cm}}
\def\xxchapitre{\hspace{.12cm} Stabilité des systèmes}


\def\xxtitreexo{Applications}%Motorisation du moteur Haibike}
\def\xxsourceexo{\hspace{.2cm} }


\def\xxposongletx{2}
\def\xxposonglettext{1.45}
\def\xxposonglety{20}
%\def\xxonglet{Part. 1 -- Ch. 3}
\def\xxonglet{Cycle 03}

\def\xxactivite{Applications}
\def\xxauteur{}%\textsl{P. Dupas.}}

\def\xxcompetences{%
\textsl{%
\textbf{Savoirs et compétences :}\\
%Les sources sont associées par un \emph{hacheur série}. La détermination des grandeurs électriques associées à ce montage permet de conclure vis à vis du cahier des charges.
%\noindent \textbf{Résoudre :} à partir des modèles retenus :
%\begin{itemize}[label=\ding{112},font=\color{ocre}] 
%\item choisir une méthode de résolution analytique, graphique, numérique;
%\item mettre en \oe{}uvre une méthode de résolution.
%\end{itemize}
%\begin{itemize}[label=\ding{112},font=\color{ocre}] 
%\item \textit{Rés -- C1.1 :} Loi entrée sortie géométrique et cinématique -- Fermeture géométrique.
%\end{itemize}
%
%\noindent \textit{Mod2 -- C4.1 :} Représentation par schéma bloc.
}}

\def\xxfigures{
%\includegraphics[width=.9\linewidth]{images/c-evolution}
}%figues de la page de garde


\def\xxpied{%
Cycle 03 -- Concevoir la commande des SLCI\\
Chapitre 1 -- \xxactivite%
}

\setcounter{secnumdepth}{5}
%---------------------------------------------------------------------------

\usepackage{pgfplots}
\begin{document}
\def\pathfig{images}
%\chapterimage{png/Fond_Cin}
\input{style/new_pagegarde}
\vspace{6cm}
\pagestyle{fancy}
\thispagestyle{plain}

\def\columnseprulecolor{\color{ocre}}
\setlength{\columnseprule}{0.4pt} 

\def\pathfig{images}

\begin{multicols}{2}

\subsection*{Correcteur proportionnel}
Soit un système de fonction de transfert $G(p)=\dfrac{1}{\left(1+10p\right)\left(1+0,1p\right)\left(1+0,2p\right)}$ placé dans une boucle à retour unitaire.

\subparagraph{}\textit{Calculer la précision du système $\varepsilon_S$ pour une entrée échelon unitaire.}

\begin{corrige}
Le système est de classe 0. L'entrée est de type échelon. $K_{\text{BO}}=1$.
L'écart statique est de $\dfrac{1}{1+1}=\dfrac{1}{2}$.
\end{corrige}

\subparagraph{}\textit{Tracer dans  le diagramme de Bode la fonction de transfert en boucle ouverte du système.}

\begin{corrige}
\begin{center}
\includegraphics[width=.9\linewidth]{images/01_Bode}
\end{center}
\end{corrige}

\subparagraph{}\textit{Déterminer $K$ pour avoir une marge de phase de 45\degres. Indiquer alors la valeur de la marge de gain. Indiquer la valeur de l'écart statique.}
\begin{corrige}
$\bullet$ On résout $\varphi\left(\omega\right)=-135\degres$ : 
$\varphi\left(\omega\right)=-\arctan 10\omega-\arctan 0,1\omega-\arctan 0,2\omega$.

$\varphi\left(\omega\right)=-135\degres \Leftrightarrow \omega = \SI{2,95}{rad.s^{-1}}$ (solveur Excel). 

$\bullet$ Calculons $G_{\text{dB}}(\omega)=-20\log\left(\sqrt{1+10^2\omega^2} \right)-20\log\left(\sqrt{1+0,1^2\omega^2} \right)-20\log\left(\sqrt{1+0,2^2\omega^2} \right)=\SI{-31}{dB}$. Il faut donc augmenter le gain de \SI{31}{dB} soit $K_P=10^{31/20}=35,48$.


$\bullet$ On a alors un écart statique de $\dfrac{1}{1+35,48}=0,027$.

$\bullet$ Pour déterminer la marge de gain, il faut résoudre $\varphi\left(\omega\right)=-180\degres$
\end{corrige}

\subparagraph{}\textit{Déterminer $K$ pour avoir une marge de gain de \SI{6}{dB}. Indiquer alors la valeur de l'écart statique.}
\begin{corrige}
\end{corrige}


\subsection*{Correcteur proportionnel} % TETE DE LECTURE
Soit un système de fonction de transfer $G(p)=\dfrac{1}{\left(1+0,05p\right)\left(1+p+2p^2\right)}$. On souhaite corriger le comportement de ce système par un correcteur proportionnel.
\subparagraph*{}\textit{Déterminer le gain $K$ qui assure une marge de phase de 45\degres.}


\subsection*{Correcteur proportionnel}
\setcounter{exo}{0}
Soit un système de fonction de transfer $G(p)=\dfrac{10}{p\left(1+p+p^2\right)}$. On souhaite corriger le comportement de ce système par un correcteur proportionnel. On désire une marge de phase de -45\degres et une marge de gain de $\SI{10}{dB}$.

\subparagraph{}\textit{Calculer la marge de phase.}
\ifprof
\begin{corrige}
On trouve $\omega=\SI{2,21}{rad/s}$ et $M_{\varphi}=-60\degres$. Le système est instable.
\end{corrige}
\else
\fi

\subparagraph{}\textit{Calculer la marge de gain.}
\ifprof
\begin{corrige}
Pour $\varphi=-180\degres$, on a $\omega=\SI{1}{rad/s}$ et $M_{G}=\SI{-20}{dB}$. Le système est instable.
\end{corrige}
\else
\fi

\subparagraph{}\textit{Déterminer $K_p$ pour avoir une marge de phase de 45\degres. Vérifier la marge de gain. }
\ifprof
\begin{corrige}
Pour $\varphi=-135\degres$ on a $\omega=\SI{0,62}{rad/s}$. On trouve un gain proportionnel de 0,54.

La marge de gain est alors de \SI{5,35}{dB} ce qui est inférieur aux \SI{10}{dB} demandés.
\end{corrige}
\else
\fi

\subparagraph{}\textit{Déterminer $K_p$ pour avoir une marge de gain de $\SI{10}{dB}$. Vérifier la marge de phase. }
\ifprof
\begin{corrige}
Pour $\varphi=-180\degres$ on a $\omega=\SI{1}{rad/s}$. On trouve un gain proportionnel de 0,316.

La marge de phase est alors de 70\degres ($\omega=\SI{0,333}{rad/s}$.
\end{corrige}
\else
\fi



\subsection*{Correcteur proportionnel intégral}
Soit un système de fonction de transfert $G(p)=\dfrac{1}{\left(p+1\right)\left(\dfrac{p}{8}+1\right)}$ placé dans une boucle à retour unitaire.

On souhaite disposer d'une marge de phase de 45\degres en utilisant un correcteur proportionnel intégral de la forme $C(p)=K_p\dfrac{1+\tau p}{\tau p}$.

\subparagraph{}\textit{Déterminer les paramètres du correcteur pour avoir une marge de phase de 45\degres. }

\subsection*{Correcteur à avance de phase}
Soit un système de fonction de transfert $G(p)=\dfrac{100}{\left(p+1\right)^2}$ placé dans une boucle à retour unitaire.
\subparagraph*{}\textit{Corriger ce système de sorte que sa marge de phase soit égale à 45\degres.}

\ifprof
\begin{itemize}
\item Pulsation de coupure à \SI{0}{dB} su système non corrigé en BO : $\omega=\SI{9,95}{rad.s^{-1}}$.
\item Pour cette pulsation la marge de phase est de 11\degres.
\item On cherche $\varphi_{\text{max}}=25\degres$ et $a=3,54$. 
\item $\omega_{\text{max}}=\SI{9,95}{rad.s^{-1}}= \dfrac{1}{T\sqrt{a}}$ ce qui conduit à $T=\SI{0,053}{s}$.
\end{itemize}
\else
\fi



\end{multicols}




\end{document}