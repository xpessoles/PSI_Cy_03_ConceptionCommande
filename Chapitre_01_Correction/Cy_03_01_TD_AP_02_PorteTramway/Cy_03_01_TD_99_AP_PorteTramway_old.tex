\documentclass[10pt,fleqn]{article} % Derfault font size and left-justified equations
\usepackage[%
    pdftitle={Correction des SLCI : Rapidité des systèmes},
    pdfauthor={Xavier Pessoles}]{hyperref}
    
\input{style/new_style}
\input{style/macros_SII}
\usepackage{multicol}
\usepackage{siunitx}
%\usepackage{picins}
\fichetrue
%\fichefalse

\proftrue
%\proffalse

\tdtrue
%\tdfalse

\courstrue
\coursfalse

\def\discipline{Sciences \\Industrielles de \\ l'Ingénieur}
\def\xxtete{Sciences Industrielles de l'Ingénieur}

\def\classe{PSI$\star$ -- MP}
\def\xxnumpartie{Cycle 03}
\def\xxpartie{Concevoir la partie commande des systèmes asservis afin de valider leurs performances}

\def\xxnumchapitre{Chapitre 1 \vspace{.2cm}}
\def\xxchapitre{\hspace{.12cm} Correction des SLCI}


\def\xxtitreexo{Système d’ouverture et de fermeture \\de portes de tramway}
\def\xxsourceexo{\hspace{.2cm} \footnotesize{Centrale Supelec -- PSI -- 2008}}


\def\xxposongletx{2}
\def\xxposonglettext{1.45}
\def\xxposonglety{20}
%\def\xxonglet{Part. 1 -- Ch. 3}
\def\xxonglet{\textsf{Cycle 03}}

\def\xxactivite{TD 99}
\def\xxauteur{\textsl{Xavier Pessoles}}

\def\xxcompetences{%
\textsl{%
\textbf{Savoirs et compétences :}\\
\footnotesize{}%Le modèle du système est donné. Des conditions sont estimées sur un paramètre de la FTBO à partir des exigences du cahier des charges. Une \textbf{démarche de réglage d'un correcteur à avance de phase} est appliquée. Une conclusion est menée en déterminant pour chaque exigence l'écart entre la performance simulée et celle attendue.}
%Les sources sont associées par un \emph{hacheur série}. La détermination des grandeurs électriques associées à ce montage permet de conclure vis à vis du cahier des charges.
%\noindent \textbf{Résoudre :} à partir des modèles retenus :
%\begin{itemize}[label=\ding{112},font=\color{ocre}] 
%\item choisir une méthode de résolution analytique, graphique, numérique;
%\item mettre en \oe{}uvre une méthode de résolution.
%\end{itemize}
%\begin{itemize}[label=\ding{112},font=\color{ocre}] 
%\item \textit{Rés -- C1.1 :} Loi entrée sortie géométrique et cinématique -- Fermeture géométrique.
%\end{itemize}
%
%\noindent \textit{Mod2 -- C4.1 :} Représentation par schéma bloc.
}}

\def\xxfigures{
\includegraphics[width=.6\linewidth]{images/fig_00}
}%figues de la page de garde


\def\xxpied{%
Cycle 03 -- Concevoir la partie commande des SLCI \\
Chapitre 1 -- \xxactivite%
}

\setcounter{secnumdepth}{5}
%---------------------------------------------------------------------------

\usepackage{pgfplots}
\begin{document}

%\chapterimage{png/Fond_Cin}
\input{style/new_pagegarde}
\vspace{5cm}
\pagestyle{fancy}
\thispagestyle{plain}

\def\columnseprulecolor{\color{ocre}}
\setlength{\columnseprule}{0.4pt} 

\def\pathfig{images}

\begin{multicols}{2}

\subsection*{Présentation}


\subsection*{Étude du régulateur de la boucle de vitesse}

\begin{obj}Déterminer un régulateur de vitesse permettant d’atteindre les exigences
suivantes :
\begin{itemize}
\item écart nul en régime permanent pour une consigne de vitesse constante et un effort perturbateur, dû à la poussée des passagers, constant;
\item marge de phase $\Delta \varphi \geq 45\degres$ pour un modèle nominal qui sera précisé par la suite;
\item bande passante la plus grande possible compte tenu de la contrainte de marge de phase;
\item temps de réponse inférieur à \SI{0,2}{s} en réponse à une variation en échelon de l’effort perturbateur.
\end{itemize}
\end{obj}

La chaîne de régulation de vitesse est décrite par le schéma-blocs suivant où la fonction de transfert représente la chaîne de mesure de vitesse comportant un filtre
du 1\ier ordre, de constante de temps $\tau_f = \SI{10}{ms}$, permettant de limiter
l’impact des bruits de mesure et $G$ est le gain de l’amplificateur de puissance
alimentant le moteur.

\begin{center}
\includegraphics[width=\linewidth]{images/fig_01}
%%\textit{}
\end{center}

On choisit d’adopter pour cette chaîne un régulateur de type proportionnel-intégral
dont la fonction de transfert est : $R(p)=K_r\left(1+\dfrac{1}{T_i p}  \right)$.

\subparagraph{}\textit{Au regard des exigences du cahier des charges, justifier le choix
de ce type de régulateur.}

On cherche d'abord à évaluer le temps de réponse vis-à-vis des perturbations. 

\subparagraph{}\textit{Déterminer la fonction de transfert en boucle fermée $T(p)=\dfrac{\Omega_1(p)}{F_1(p)}$ entre les perturbations dues à la poussée des passagers et la vitesse du moteur, en fonction des différentes fonctions de transfert de la figure précédente.  Montrer que la réponse fréquentielle peut être approchée par la relation :}

$$
|| T\left(j \omega\right) || = 
|| H_2\left(j \omega\right) || \cdot 
\min \left(
||H_1\left(j \omega\right)||; 
\left|\left| \dfrac{1}{R\left(j \omega\right)GH_3\left(j \omega\right)}\right|\right|\right)
$$

$$
= || H_2\left(j \omega\right) || || M\left(j \omega\right) ||.
$$


Pour la suite, on adopte les modèles de commande simplifiés suivants : 
$$
H_1(p)=\dfrac{10}{p} \quad
H_2(p)=0,05 \quad
H_3(p)=\dfrac{0,1}{1+0,01 p}\quad
G=10.
$$

Afin de limiter le périmètre de l’étude, on adopte sans justification les
hypothèses suivantes : 
\begin{itemize}
\item $1/T_i < \SI{100}{rad.s^{-1}}$;
\item la situation considérée est celle de la figure suivante représentant le diagramme asymptotique de la fonction 
$
\left|\left|\dfrac{1}{R\left(j \omega\right)GH_3\left(j \omega\right)}\right|\right|_{\text{dB}}
$ où $20\log G_0 < 0$.
\end{itemize}

\begin{center}
\includegraphics[width=\linewidth]{images/fig_02}
%%\textit{}
\end{center}

\subparagraph{}\textit{Exprimer $G_0$ en fonction de $K_r$. En utilisant la figure précédente, tracer le diagramme asymptotique de la fonction $\left|\left|H_1\left(j \omega\right)\right|\right|$ (veiller
au respect des pentes) et celui de $\left|\left|M\left(j \omega\right)\right|\right|$ en adoptant l’approximation de la
question précédente.}

\subparagraph{}\textit{En déduire alors une approximation de la fonction de transfert
$T(p)=\dfrac{\Omega_1(p)}{F_1(p)}$ en exprimant toutes les brisures en fonction de $K_r$ et $T_i$.}

\subparagraph{}\textit{Proposer une nouvelle expression approchée de $T(p)$ sous la forme $T_a(p)=\dfrac{N(p)}{1+\tau p}$ où $N(p)$ est le numérateur de $T(p)$? En utilisant la forme approchée de $T_a(p)$, déterminer l'évolution de la vitesse $\Omega_1(t)$ en réponse à un échelon de la force de perturbation et tracer son allure. }

\subparagraph{}\textit{En se référant à des fonctions types connues donner, en fonction de $T_i$, un ordre de grandeur du temps de réponse vis-à-vis de la force perturbatrice.}

\subparagraph{}\textit{Justifier alors l’intérêt d’adopter pour $T_i$ la valeur la plus petite possible.}

\subparagraph{}\textit{En vous aidant de tracés succincts de diagrammes
de Bode, analyser la stabilité du système
bouclé dans les deux cas : $\dfrac{1}{T_i}>\SI{100}{rad.s^{-1}}$ et $\dfrac{1}{T_i}<\SI{100}{rad.s^{-1}}$.}

\subparagraph{}\textit{En prenant $K_r=1$, tracer les diagrammes
de Bode asymptotiques (module et phase) de la fonction de transfert en boucle
ouverte corrigée et l’allure de la courbe réelle du diagramme de phase. Veiller à
effectuer ce tracé de façon à respecter une situation stable du système en boucle
fermée.}

\subparagraph{}\textit{En utilisant la représentation dans le plan de Bode donnée figure suivante, déterminer
quelle est la valeur $T_{i\text{min}}$ la plus petite possible que l’on peut conférer à $T_i$
compatible avec la marge de phase minimale exigée par le cahier des charges
(cette fonction servira uniquement à calculer en plaçant judicieusement
pour obtenir la marge de phase souhaitée).}


\begin{center}
\includegraphics[width=\linewidth]{images/fig_03}
%%\textit{}
\end{center}

\subparagraph{}\textit{En conservant la valeur $T_{i\text{min}}$ calculée précédemment, en déduire alors la
valeur du gain $K_r$ du régulateur permettant d’assurer la marge de phase souhaitée.}

\subparagraph{}\textit{Vérifier si le cahier des charges est validé, et conclure sur l’adéquation
du régulateur calculé vis-à-vis du problème posé.}
%
%\subparagraph{}\textit{}
%
%\subparagraph{}\textit{}

\end{multicols}

%
%\begin{center}
%\includegraphics[width=\linewidth]{images/fig_07}
%%\textit{}
%\end{center}


\end{document}

\subparagraph{}\textit{}
\begin{center}
\includegraphics[width=\linewidth]{images/fig_07}
%\textit{}
\end{center}


%\newpage

%\begin{center}
%\includegraphics[width=\linewidth]{images/cor_01}
%\textit{}
%\end{center}






\begin{center}
\includegraphics[width=\linewidth]{images/fig_06}
%\textit{}
\end{center}
\begin{center}
\includegraphics[width=\linewidth]{images/img_04}
%\textit{}
\end{center}

